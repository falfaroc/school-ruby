\documentclass{exam}
\begin{document}
FIRST QUIZ: CS4472A Tuesday, 3 October 2017, 7:10pm, Room MC17\newline
\newline
\newline
\newline
NAME AS APPEARS ON STUDENT ID:\newline
\newline
STUDENT ID NUMBER:\newline
\newline
UWO/CONFLUENCE USER NAME:\newline
\newline
REMINDERS:
\begin{enumerate}
\item (from course outline) The quiz will be closed book, closed notes, with no electronic devices allowed, with particular reference to any electronic devices that are capable of communication and/or storing information.
\item Write neatly.  If the marker can't read it, it is wrong.
\item This exam shouldn't take long to write.  On the other hand, time will pass.  It is a 30 minute quiz with 20 questions.  If you complete a question every  minute you will still have 10 minutes at the end to double check that everything is in order.
\item While you are not allowed to open the exam booklet until the proctor says you can, you can fill out the information on the cover page.  You should also get out your student id and make sure your pencils and pens are in order.  If you need to get something out of your jacket or knapsack once the exam has started, raise your hand and wait til a proctor comes to you to oversee the matter.
\end{enumerate}
\newpage
\begin{enumerate}
\item The Capability Maturity Model for US government contractors distinguishes 5 levels of company software development process.  Level 5 is characterized as ANSWER\newline
ANSWER=
\item The D in SOLID stands for ANSWER\newline
ANSWER=
\item The scripts that were designed to aid the practice process assume that you will be uploading a copy of your work to BitBucket every time you ANSWER\newline
ANSWER=
\item Although we often think of programs as taking inputs and producing outputs, a higher level view of what is going on is to think of the programs as ANSWER about how to take inputs and produce outputs.\newline
ANSWER=
\item The per cent of the total mark allocated for all the quizzes is ANSWER\newline
ANSWER=
\item Using combinatorial testing, if I have 10 binary inputs, I only need to use ANSWER test cases (each a setting of each of the 10 inputs) to expect to find 98 per cent of the errors in the program.\newline
ANSWER=
\item The number of practice reviews that CS4472 will have this semester is ANSWER\newline
ANSWER=
\item The first testing framework for Ruby that we are looking at is called ANSWER\newline
ANSWER=
\item When multiple methods of a class have the same parameters, this is a code smell called ANSWER\newline
ANSWER=
\item The differences between RSpec and Cucumber result from the intent that Cucumber test files are meant to be readable by ANSWER\newline
ANSWER=
\item Many of the ideas of the Capability Maturity Model were adapted to individual developers under the name ANSWER\newline
ANSWER=
\item MicroTest (MiniTest subset) discourages the writing of tests that depend on side-effects of the previous test by ANSWER\newline
ANSWER=
\item The protocols for practice expect that the longest amount of time that you will practice before recording a note is ANSWER\newline
ANSWER=
\item The testing technique called boundary value partition starts with the notion of breaking the space of inputs into ANSWER\newline
ANSWER=
\item The corporate policy of developers merging their working copies into the main line of the branch repository several times a day is called ANSWER\newline
ANSWER=
\item The motivation behind multiple merges per day per developer is to ANSWER\newline
ANSWER=
\item In MiniTest, we write test classes that inherit from Test, but in RSpec these test classes are actually being created at runtime by ANSWER\newline
ANSWER=
\item A study by Ahmed et al found that the probability of errors in untested code was ANSWER the probability of errors in tested code\newline
ANSWER=
\item The first tool for checking code quality for programs written in Ruby is ANSWER, which is described as a code smell detector.\newline
ANSWER=
\item The kind of testing we do to make sure that when we change a program we do not break something that used to work is called ANSWER\newline
ANSWER=
\end{enumerate}
\newpage
\begin{verbatim}
exam_database_file= examdatabase.json
exam_format= latex
dump_database= false
line_width= 72
question_count= 20
create_exam= false
answer_key= true
sample_seed= 2322
shuffle_seed= 245
["C1", "C2", "C3", "C4", "C5", "C6"]
["C1", "C2", "C3", "C4", "C5", "C6"]
\end{verbatim}
\begin{enumerate}
\item continually improving
\item dependency inversion principle
\item record a note about your practice progress
\item encode knowledge
\item 21
\item 13
\item 4
\item minitest
\item data clumping
\item the customer and the programmer
\item Personal Software Process
\item running tests in random order
\item 30 minutes
\item regions of interest
\item continuous integration
\item minimize merge conflicts
\item describe
\item twice
\item reek
\item regression testing
\end{enumerate}
\end{document}

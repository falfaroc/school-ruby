\documentclass{exam}
\begin{document}
FIRST QUIZ: CS4472A Tuesday, 3 October 2017, 7:10pm, Room MC17\newline
\newline
\newline
\newline
NAME AS APPEARS ON STUDENT ID:\newline
\newline
STUDENT ID NUMBER:\newline
\newline
UWO/CONFLUENCE USER NAME:\newline
\newline
REMINDERS:
\begin{enumerate}
\item (from course outline) The quiz will be closed book, closed notes, with no electronic devices allowed, with particular reference to any electronic devices that are capable of communication and/or storing information.
\item Write neatly.  If the marker can't read it, it is wrong.
\item This exam shouldn't take long to write.  On the other hand, time will pass.  It is a 30 minute quiz with 20 questions.  If you complete a question every  minute you will still have 10 minutes at the end to double check that everything is in order.
\item While you are not allowed to open the exam booklet until the proctor says you can, you can fill out the information on the cover page.  You should also get out your student id and make sure your pencils and pens are in order.  If you need to get something out of your jacket or knapsack once the exam has started, raise your hand and wait til a proctor comes to you to oversee the matter.
\end{enumerate}
\newpage
\begin{enumerate}
\item The per cent of the total mark allocated for all the practice reviews is ANSWER
\begin{itemize}
\item 49
\end{itemize}
\item The per cent of the total mark allocated for all the weekly practices is ANSWER
\begin{itemize}
\item 30
\end{itemize}
\item To illustrate the relation between testing and software design, we will look at the programming technique ANSWER
\begin{itemize}
\item test driven development
\end{itemize}
\item The practice technique advocated in this class is a modification of the ANSWER
\begin{itemize}
\item Personal Software Process
\end{itemize}
\item The per cent of the total mark allocated for all the quizzes is ANSWER
\begin{itemize}
\item 21
\end{itemize}
\item A main theme behind the practice technique advocated in this class is that in order to improve your programming, ANSWER
\begin{itemize}
\item you need data about your past programming
\end{itemize}
\item The first tool for checking code quality for programs written in Ruby is ANSWER, which is described as a code smell detector.
\begin{itemize}
\item reek
\end{itemize}
\item The first testing framework for Ruby that we are looking at is called ANSWER
\begin{itemize}
\item minitest
\end{itemize}
\item It is easy to make up test inputs, but it can be tricky to know what the right output for a given input should be.  This is refered to as the ANSWER problem
\begin{itemize}
\item Oracle
\end{itemize}
\item The testing technique called boundary value partition starts with the notion of breaking the space of inputs into ANSWER
\begin{itemize}
\item regions of interest
\end{itemize}
\item An important concept we will look at related to the question of when has one done enough testing is ANSWER
\begin{itemize}
\item coverage
\item mutation
\end{itemize}
\item Testing is generally about finding errors that have already been made.  This course also covers the topic of ANSWER, which is about trying to prevent errors from being made in the first place.
\begin{itemize}
\item quality assurance
\end{itemize}
\item The protocols for practice expect that the longest amount of time that you will practice before recording a note is ANSWER
\begin{itemize}
\item 30 minutes
\end{itemize}
\item The number of weekly practices that CS4472 will have this semester is ANSWER
\begin{itemize}
\item 10
\end{itemize}
\item A common piece of information for people interested in programmer productivity to track is ANSWER
\begin{itemize}
\item time spent
\item number of lines of code written
\item number of defects found
\end{itemize}
\item The scripts that were designed to aid the practice process assume that you will be uploading a copy of your work to BitBucket every time you ANSWER
\begin{itemize}
\item record a note about your practice progress
\end{itemize}
\item The number of practice reviews that CS4472 will have this semester is ANSWER
\begin{itemize}
\item 4
\end{itemize}
\item The four phases of testing (according to Whittaker) are: 1) modeling the software environment, 2) selecting test cases, 3) running and checking test cases, and 4) ANSWER
\begin{itemize}
\item checking how well the testing is going
\end{itemize}
\item The total amount of practice time you can get credit for during a practice week is ANSWER
\begin{itemize}
\item 3 hours
\end{itemize}
\item The kind of testing we do to make sure that when we change a program we do not break something that used to work is called ANSWER
\begin{itemize}
\item regression testing
\end{itemize}
\end{enumerate}
\end{document}

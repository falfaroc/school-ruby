\documentclass{exam}
\begin{document}
FIRST QUIZ: CS4472A Tuesday, 3 October 2017, 7:10pm, Room MC17\newline
\newline
\newline
\newline
NAME AS APPEARS ON STUDENT ID:\newline
\newline
STUDENT ID NUMBER:\newline
\newline
UWO/CONFLUENCE USER NAME:\newline
\newline
REMINDERS:
\begin{enumerate}
\item (from course outline) The quiz will be closed book, closed notes, with no electronic devices allowed, with particular reference to any electronic devices that are capable of communication and/or storing information.
\item Write neatly.  If the marker can't read it, it is wrong.
\item This exam shouldn't take long to write.  On the other hand, time will pass.  It is a 30 minute quiz with 20 questions.  If you complete a question every  minute you will still have 10 minutes at the end to double check that everything is in order.
\item While you are not allowed to open the exam booklet until the proctor says you can, you can fill out the information on the cover page.  You should also get out your student id and make sure your pencils and pens are in order.  If you need to get something out of your jacket or knapsack once the exam has started, raise your hand and wait til a proctor comes to you to oversee the matter.
\end{enumerate}
\newpage
\begin{enumerate}
\item Although we often think of programs as taking inputs and producing outputs, a higher level view of what is going on is to think of the programs as ANSWER about how to take inputs and produce outputs.\newline
ANSWER=
\item The scripts that were designed to aid the practice process assume that you will be uploading a copy of your work to BitBucket every time you ANSWER\newline
ANSWER=
\item To illustrate the relation between testing and software design, we will look at the programming technique ANSWER\newline
ANSWER=
\item The per cent of the total mark allocated for all the practice reviews is ANSWER\newline
ANSWER=
\item The per cent of the total mark allocated for all the quizzes is ANSWER\newline
ANSWER=
\item A main theme behind the practice technique advocated in this class is that in order to improve your programming, ANSWER\newline
ANSWER=
\item The first testing framework for Ruby that we are looking at is called ANSWER\newline
ANSWER=
\item The per cent of the total mark allocated for all the quizzes is ANSWER\newline
ANSWER=
\item q23\newline
ANSWER=
\item The number of quizzes CS4472 will have this semester is ANSWER\newline
ANSWER=
\item q27\newline
ANSWER=
\item The per cent of the total mark allocated for all the weekly practices is ANSWER\newline
ANSWER=
\item The protocols for practice expect that the longest amount of time that you will practice before recording a note is ANSWER\newline
ANSWER=
\item q28\newline
ANSWER=
\item The practice technique advocated in this class is a modification of the ANSWER\newline
ANSWER=
\item An important concept we will look at related to the question of when has one done enough testing is ANSWER\newline
ANSWER=
\item q26\newline
ANSWER=
\item q30\newline
ANSWER=
\item Testing is generally about finding errors that have already been made.  This course also covers the topic of ANSWER, which is about trying to prevent errors from being made in the first place.\newline
ANSWER=
\item q24\newline
ANSWER=
\end{enumerate}
\newpage
\begin{verbatim}
exam_database_file= examdatabase.json
exam_format= latex
dump_database= false
line_width= 72
question_count= 20
create_exam= false
answer_key= true
sample_seed= 2322
shuffle_seed= 245
["C1", "C2", "C3"]
["C1", "C2", "C3"]
\end{verbatim}
\begin{enumerate}
\item encode knowledge
\item record a note about your practice progress
\item test driven development
\item 49
\item 21
\item you need data about your past programming
\item minitest
\item 21
\item a23
\item 3
\item a27
\item 30
\item 30 minutes
\item a28
\item Personal Software Process
\item \begin{itemize}
\item coverage
\item mutation
\end{itemize}
\item a26
\item a30
\item quality assurance
\item a24
\end{enumerate}
\end{document}

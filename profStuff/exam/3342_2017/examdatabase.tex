\documentclass{exam}
\begin{document}
FINAL EXAM: CS3342b Tuesday, 25 April 2017, 2pm, Room FEB GYM\newline
\newline
\newline
\newline
NAME AS APPEARS ON STUDENT ID:\newline
\newline
STUDENT ID NUMBER:\newline
\newline
GAUL/CONFLUENCE USER NAME:\newline
\newline
REMINDERS:
\begin{enumerate}
\item (from course outline) The final exam will be closed book, closed notes, with no electronic devices allowed, with particular reference to any electronic devices that are capable of communication and/or storing information.
\item Write neatly.  If the marker can't read it, it is wrong.
\item This exam shouldn't take long to write.  On the other hand, time will pass.  It is a 3 hour exam with 50 questions.  If you complete a question every 3 minutes (or 10 questions every half hour), you will still have a half hour at the end to double check that everything is in order.
\item While you are not allowed to open the exam booklet until the proctor says you can, you can fill out the information on the cover page.  You should also get out your student id and make sure your pencils and pens are in order.  If you need to get something out of your jacket or knapsack once the exam has started, raise your hand and wait til a proctor comes to you to oversee the matter.
\end{enumerate}
\newpage
\begin{enumerate}
\item Adding a feature to a programming language to make it easier to do something that was already doable is called adding ANSWER.
\begin{itemize}
\item syntactic sugar
\end{itemize}
\item Matz, the creator of Ruby, thinks that it is less important to optimize the execution (efficiency) of a programming language and more important to optimize the efficiency of ANSWER.
\begin{itemize}
\item the programmers
\end{itemize}
\item A programming language is called ANSWER, if it is executed directly by an interpreter rather than by first being compiled with a compiler.
\begin{itemize}
\item interpreted
\end{itemize}
\item If the types of a programming language are bound at execution time rather than compile time, then the types are called ANSWER.
\begin{itemize}
\item dynamically typed
\end{itemize}
\item In describing the properties of an object-oriented language, encapsulation means ANSWER.
\begin{itemize}
\item data and behavior are packaged together
\item there is a mechanism for restricting access to an object's components
\end{itemize}
\item In discussing object-oriented languages, objects are organized into a class tree to support the property of ANSWER.
\begin{itemize}
\item inheritance
\end{itemize}
\item In discussing object-oriented languages, being able to handle objects of related types is called ANSWER.
\begin{itemize}
\item polymorphism
\end{itemize}
\item The application that caused a significant increase in the popularity of Ruby was a web framework called ANSWER.
\begin{itemize}
\item Rails
\item Ruby on Rails
\end{itemize}
\item The main concurrency approach used in Ruby is ANSWER.
\begin{itemize}
\item threads
\end{itemize}
\item The command name for the Ruby interpreter is ANSWER.
\begin{itemize}
\item irb
\end{itemize}
\item In Ruby, true.class returns ANSWER.
\begin{itemize}
\item TrueClass
\end{itemize}
\item Ruby supports two common ways that boolean expressions are handled in programming languages.  In one approach, both subexpressions of a boolean operator are evaluated before the boolean operator is evaluated.  In the other approach, called ANSWER, the first subexpression in a boolean expression is evaluated and, if that is enough to know the result of the boolean expression, then the second subexpression is not evaluated.
\begin{itemize}
\item short-circuit evaluation
\end{itemize}
\item In Ruby, normally, when you try to add a String to a Fixnum, you get an error message saying that a String can't be coerced to a Fixnum.  This is because Ruby is ANSWER typed.
\begin{itemize}
\item strongly
\end{itemize}
\item One way of checking types is to see what constructor was used to create an object that is a parameter.  Another way of checking types is to wait until a method is sent to an object and see if it supports the method.  This second way is called ANSWER
\begin{itemize}
\item duck typing
\end{itemize}
\item A major claim in object-oriented design philosophy is that you should code to ANSWER rather than code to the implementation.
\begin{itemize}
\item interface
\end{itemize}
\item The \& notation in the line of Ruby def george(\&sam) is used to indicate that sam is ANSWER.
\begin{itemize}
\item a code block
\end{itemize}
\item The : notation in the Ruby expressions :hi is used to indicate that hi is ANSWER.
\begin{itemize}
\item a symbol
\end{itemize}
\item With respect to the value returned by the Ruby expression 'hi'.object\_id == 'hi'.object\_id, you can say it ANSWER.
\begin{itemize}
\item could be either true or false
\end{itemize}
\item With respect to the value returned by the Ruby expression :hi.object\_id == :hi.object\_id, you can say it ANSWER.
\begin{itemize}
\item will always be true
\end{itemize}
\item To execute a code block in Ruby that is passed to a method but doesn't appear on its parameter list, you use the keyword ANSWER.
\begin{itemize}
\item yield
\end{itemize}
\item To execute a code block in Ruby that is passed to a method on its parameter list, you send that parameter the method ANSWER.
\begin{itemize}
\item call
\end{itemize}
\item A code block in Ruby is some lines of code surrounded by either curly braces or ANSWER.
\begin{itemize}
\item do end
\end{itemize}
\item In Ruby, the expression Fixnum.class returns ANSWER.
\begin{itemize}
\item Class
\end{itemize}
\item The root of the inheritance hierarchy in Ruby is the class ANSWER.
\begin{itemize}
\item Object
\end{itemize}
\item In Ruby, the name of the method in the class Me that is automatically invoked when a new object of type Me is created with Me.new is ANSWER.
\begin{itemize}
\item initialize
\end{itemize}
\item In Ruby, the @ is used to indicate that the variable @me is ANSWER.
\begin{itemize}
\item an instance variable
\end{itemize}
\item In Ruby, the @@ is used to indicate that the variable @@me is ANSWER.
\begin{itemize}
\item a class variable
\end{itemize}
\item In Ruby, by convention, the ? in the method me? is used to indicate that me is ANSWER.
\begin{itemize}
\item boolean
\end{itemize}
\item In Ruby, the mixin is used to solve the object-oriented programming problem of ANSWER.
\begin{itemize}
\item multiple inheritance
\end{itemize}
\item The feature of programs being able to `write programs' (creating application specific language features) is called ANSWER.
\begin{itemize}
\item metaprogramming
\end{itemize}
\item In Ruby, if you declare a class with a class name that is already in use and put in it the definition of a new method, you have changed the functionality of the existing class (even if it is a predefined class like Fixnum).  The property of Ruby that allows this is ANSWER.
\begin{itemize}
\item open classes
\end{itemize}
\item When you send a message to a Ruby object, Ruby first looks at the methods that object supports, and then starts working the inheritance chain.  If it still can't find the appropriate method, the message and its parameters get passed as a message to the original object looking for a method called ANSWER.
\begin{itemize}
\item method\_missing
\end{itemize}
\item In the Ruby community, the acronym DSL is an abbreviation for ANSWER.
\begin{itemize}
\item domain specific language
\end{itemize}
\item In Ruby, if a line starts with a method name, that method is being sent to the object named ANSWER.
\begin{itemize}
\item self
\end{itemize}
\item When you define a method in a class, normally it is meant to be invoked on an object of that class (an instance method).  Sometimes it is meant to be invoked on the class name itself (a class method), like Date.parse('3rd Feb 2001').  In Ruby, to define a class method, we put ANSWER at the beginning of the method name in its definition.
\begin{itemize}
\item self.
\end{itemize}
\item Instead of the +  symbol, Haskell uses the symbol ANSWER for a string concatenation operator.
\begin{itemize}
\item ++
\end{itemize}
\item The type of a string constant in Haskell, by default, is written ANSWER.
\begin{itemize}
\item $\lbrack$Char$\rbrack$
\end{itemize}
\item In Haskell, you use the keyword ANSWER to collect related code into a similar scope.
\begin{itemize}
\item module
\end{itemize}
\item In Haskell, if I define a function double x = x + x, its type signature would be ANSWER.
\begin{itemize}
\item $(Num \quad a) \Rightarrow a \rightarrow a$
\end{itemize}
\item In Haskell, instead of writing something like if x == 0 then 1 else fact ( x - 1 ) * x, you can write a series of lines starting with factorial 0 = 1.  This second style is called ANSWER.
\begin{itemize}
\item pattern matching
\end{itemize}
\item In Haskell, instead of writing something like if x == 0 then 1 else fact ( x - 1 ) * x, you can write a series of lines starting with $|$ x $>$ 1 = x * factorial ( x - a).  This second style is called ANSWER.
\begin{itemize}
\item using guards
\end{itemize}
\item In Haskell, instead of defining second by writing something like second x = head( tail(  x ) ), you can write this without introducing the parameter x by using function composition.  Doing that, you would define second by ANSWER.
\begin{itemize}
\item second = head . tail
\end{itemize}
\item In Haskell, if I write (h:t) = $\lbrack$3, 5, 7$\rbrack$, ANSWER is the value of h.
\begin{itemize}
\item 3
\end{itemize}
\item In Haskell, if I write (h:t) = $\lbrack$3, 5, 7$\rbrack$, ANSWER is the value of t.
\begin{itemize}
\item $\lbrack$5, 7$\rbrack$
\end{itemize}
\item In Haskell, ANSWER is the output of zip $\lbrack$17..20$\rbrack$ $\lbrack$10,8..4$\rbrack$.
\begin{itemize}
\item $\lbrack$(17,10),(18,8),(19,6),(20,4)$\rbrack$
\end{itemize}
\item In Haskell, ANSWER is the output of zip $\lbrack$20..17\rbrack$ $\lbrack$10,8..4\rbrack$.
\begin{itemize}
\item $\lbrack\rbrack$
\end{itemize}
\item In Haskell, defining lists using a notation like $\lbrack x * 2 | x \leftarrow \lbrack 3, 4, 5\rbrack \rbrack$ is called using ANSWER.
\begin{itemize}
\item list comprehensions
\end{itemize}
\item In Haskell, $\lbrack x * 2 | x \leftarrow \lbrack 3, 4, 5\rbrack \rbrack$  evaluates to ANSWER.
\begin{itemize}
\item $\lbrack 6, 8, 10\rbrack$
\end{itemize}
\item In Haskell, the anonymous function ANSWER causes the expression `map ANSWER $\lbrack 1, 2, 3\rbrack$' to produce $\lbrack -4, -5, -6\rbrack$.
\begin{itemize}
\item $(\setminus x \rightarrow - (x + 3))$
\end{itemize}
\item In Haskell, if we want to define a local named function inside a function definition, we use the keyword ANSWER.
\begin{itemize}
\item where
\end{itemize}
\item In Haskell, the type signature of the function sum x y = x + y is ANSWER.
\begin{itemize}
\item $(Num \quad a) => a \rightarrow a \rightarrow a$
\end{itemize}
\item In Haskell, given the definition sum x y = x + y, ANSWER is the value of that is produced by the expression (sum 3).
\begin{itemize}
\item $(\setminus x \rightarrow 3 + x)$
\end{itemize}
\item The way Haskell handles functions with more than one parameter is called ANSWER.
\begin{itemize}
\item currying
\end{itemize}
\item In most languages, a function definition like f a b = a : (f (a + b) b) would result in an infinite recursion.  However, in Haskell we can partially evaluate functions like this because Haskell is based on ANSWER.
\begin{itemize}
\item lazy evaluation
\end{itemize}
\item Although Haskell is a statically typed language, we usually don't need to write type declarations because Haskell uses ANSWER to figure out what the types are.
\begin{itemize}
\item type inference
\end{itemize}
\item In Haskell, we can declare the type of a parameter to a function to be something specific like Char.  However, we can also declare the type of a parameter to be something that could include many types like ListLike that supports the functions head and tail.  We do this with a definition of ListLike that begins with the keyword ANSWER.
\begin{itemize}
\item class
\end{itemize}
\item One of the three most significant parts of a monad is called ANSWER, which wraps up a function and puts it in the monad's container.
\begin{itemize}
\item return
\end{itemize}
\item One of the three most significant parts of a Haskell monad is called ANSWER, which unwraps a function.
\begin{itemize}
\item $>>=$
\item a bind function
\end{itemize}
\item In Haskell's do notation for working with monads, assignment uses the ANSWER operator.
\begin{itemize}
\item $\leftarrow$
\end{itemize}
\item Since Haskell doesn't have traditional error handling mechanisms, by convention, people use the ANSWER monad to distinguish a valid return from an error return.
\begin{itemize}
\item Maybe
\end{itemize}
\item When viewing programming languages as natural languages, the word ANSWER is used instead of `words'.
\begin{itemize}
\item tokens
\end{itemize}
\item The routine in a compiler that takes as input a sequence of characters outputs these characters grouped into meaningful units is called ANSWER.
\begin{itemize}
\item a lexical analyzer
\item a scanner
\item a lexer
\end{itemize}
\item The specifications for how to group characters into meaningful units are traditionally written as ANSWER.
\begin{itemize}
\item regular expressions
\end{itemize}
\item The specifications of how to group characters into meaningful basic units of a programming language are generally implemented in code that has the abstract form of ANSWER.
\begin{itemize}
\item a finite automata
\item a finite state machine
\end{itemize}
\item When viewed formally, a language is defined as a set of ANSWER.
\begin{itemize}
\item strings
\end{itemize}
\item The Greek letter epsilon, when talking about languages, is used to represent ANSWER.
\begin{itemize}
\item the empty string
\end{itemize}
\item In automatically generating the code that reads characters and outputs the part of a programming language that is analogous to its words, we start with a specification and then traditionally convert it into code in two stages.  In the first stage, we produce ANSWER.
\begin{itemize}
\item a nondeterministic finite automata
\item a nondeterministic finite state machine
\end{itemize}
\item In automatically generating the code that reads characters and outputs the part of a programming language that is analogous to its words, we start with a specification and then traditionally convert it into code in two stages.  The main problem that can arise in moving from the first stage to the second stage is ANSWER.
\begin{itemize}
\item an exponential explosion in the number of states needed
\end{itemize}
\item Three concepts related to concurrency were discussed with regards to the language Io.  ANSWER was presented as a way to manage two execution streams that pass control back and forth between themselves.
\begin{itemize}
\item coroutines
\end{itemize}
\item Three concepts related to concurrency were discussed with regards to the language Io.  ANSWER was presented as a general mechanism for sending a message to an object that would cause that object to respond to the message as a separate process running asynchronously.
\begin{itemize}
\item Actors
\end{itemize}
\item Three concepts related to concurrency were discussed with regards to the language Io.  ANSWER was presented as a way to request that something be computed and then be able to continue computing until the result was needed.  If the result was available, then things would proceed as expected.  If the result was not available, then a wait would be initiated until the result became available.
\begin{itemize}
\item Futures
\end{itemize}
\item Io is known for taking ANSWER -based approach to object-oriented programming.
\begin{itemize}
\item a prototype
\end{itemize}
\item In Io, the basic method for creating a new object is ANSWER.
\begin{itemize}
\item clone
\end{itemize}
\item In Io, the type of an object is generally the nearest ancestor that ANSWER.
\begin{itemize}
\item has a name that starts with a capital letter
\item has a slot for the method type
\end{itemize}
\item In Io, we create a singleton by redefining the method ANSWER.
\begin{itemize}
\item clone
\end{itemize}
\item In Ruby, the evaluation of arguments to a message are handled by the object sending the message.  In Haskell, the runtime environment decides when and how much to evaluate an argument to a function.  In Io, the evaluation of the arguments to a message is made by ANSWER.
\begin{itemize}
\item the reciever of the message
\end{itemize}
\item In Io, a message has three aspects that can be interrogated by the call method.  They are: the sender, the reciever, and ANSWER.
\begin{itemize}
\item the argument list
\end{itemize}
\item Io allows programmers to play with its syntax, doing things like introducing a colon operator and redefining how curly braces are processed.  This makes it easy to use Io to create ANSWER.
\begin{itemize}
\item Domain Specific Languages
\item DSLs
\end{itemize}
\item As one would expect in an object-oriented language, when a message is sent to an object, the first thing the system does is to look for a corresponding method in that object.  However, Io lets you change what happens next by redefining the method named ANSWER.
\begin{itemize}
\item forward
\end{itemize}
\item The central idea of context-free grammars is to define a language by productions.  These productions say that a nonterminal symbol can be replaced by ANSWER.
\begin{itemize}
\item a sequence of terminals and nonterminals
\item a sequence of symbols
\end{itemize}
\item The specific type of context-free grammar that was the main focus of the portion of the Syntax Analysis chapter that was assigned was ANSWER.
\begin{itemize}
\item LL(1)
\end{itemize}
\item In a context-free grammar, the nonterminal that derives an entire member of the language being defined is called ANSWER.
\begin{itemize}
\item a start symbol
\end{itemize}
\item Using the context-free grammar based on the two rules $A \rightarrow b A$ and $A \rightarrow b$, ANSWER would be the derivation sequence for bbb.
\begin{itemize}
\item $A \Rightarrow Ab \Rightarrow Abb \Rightarrow bbb$
\end{itemize}
\item ANSWER is the regular expression that corresponds to the language defined by the context-free grammar with the three rules $A \rightarrow A a$, $A \rightarrow A b$, $A \rightarrow a$.
\begin{itemize}
\item $a (a|b)*$
\end{itemize}
\item ANSWER would be the derivation of ((1)) in the language defined by the context-free grammar consisting of the two rules $E \rightarrow ( E )$ and $E \rightarrow 1$.
\begin{itemize}
\item $E \Rightarrow (E) \Rightarrow ((E)) \Rightarrow ((1))$
\end{itemize}
\item ANSWER are two derivations of the string cc that produce distinct syntax trees from the context-free grammar $X \rightarrow X c Y$ , $Y \rightarrow X$ ,  $Y \rightarrow$ and $X \rightarrow$ .
\begin{itemize}
\item $X \Rightarrow XcY \Rightarrow XcYcY \Rightarrow cYcY \Rightarrow ccY \Rightarrow cc$ AND $X \Rightarrow XcY \Rightarrow XcX \Rightarrow XcXcY \Rightarrow cXcY \Rightarrow ccY \Rightarrow cc$
\end{itemize}
\item When a grammar can produce two distinct syntax trees for the same string, the grammar is said to be ANSWER.
\begin{itemize}
\item ambiguous
\end{itemize}
\item If I wanted to fix the grammar $E \rightarrow E + E$ and $E \rightarrow id$, so that it would only produce one syntax tree, which is left recursive; the new grammar would be ANSWER.
\begin{itemize}
\item $E \rightarrow E + F$ and $E \rightarrow F$ and $F \rightarrow id$
\item $E \rightarrow E + F$ and $E \rightarrow id$ and $F \rightarrow id$
\end{itemize}
\item One aspect of the if-then-else-end syntax of Ruby is that it avoids the ANSWER problem.
\begin{itemize}
\item dangling else
\end{itemize}
\item In the context-free grammar $A \rightarrow B A$ , $B \rightarrow A B$, $A \rightarrow B$, $A \rightarrow a$, $B \rightarrow b$, and $B \rightarrow$  the value of Nullable(A) is ANSWER.
\begin{itemize}
\item true
\end{itemize}
\item In the context-free grammar $A \rightarrow B A$ , $B \rightarrow A B$, $A \rightarrow a$, $B \rightarrow b$, $B \rightarrow$  the value of Nullable(A) is ANSWER.
\begin{itemize}
\item false
\end{itemize}
\item In the context-free grammar $A \rightarrow B A$ , $B \rightarrow A B$, $A \rightarrow B$, $A \rightarrow a$, $B \rightarrow b$, and $B \rightarrow$  the value of FIRST(A) is ANSWER.
\begin{itemize}
\item $\{a,b\}$
\end{itemize}
\item In the context-free grammar $A \rightarrow B A$ , $B \rightarrow A B$, $A \rightarrow a$, $B \rightarrow b$, $B \rightarrow$  the value of FIRST(A) is ANSWER.
\begin{itemize}
\item $\{a,b\}$
\end{itemize}
\item In the context-free grammar $A \rightarrow B A$ , $B \rightarrow A B$, $A \rightarrow B$, $A \rightarrow a$, $B \rightarrow b$, and $B \rightarrow$  the value of FOLLOW(A) is ANSWER.
\begin{itemize}
\item $\{a,b\}$
\end{itemize}
\item In the context-free grammar $A \rightarrow B A$ , $B \rightarrow A B$, $A \rightarrow a$, $B \rightarrow b$, $B \rightarrow$  the value of FOLLOW(A) is ANSWER.
\begin{itemize}
\item $\{b\}$
\end{itemize}
\item The context-free grammar $A \rightarrow B A$ , $B \rightarrow A B$, $A \rightarrow a$, $B \rightarrow b$, $B \rightarrow$  is not LL(1) specifically because ANSWER.
\begin{itemize}
\item FIRST(BA) and FIRST(a) both include a, so we do not know which A rule to use
\end{itemize}
\item When you write a parser for a context-free grammar that satisfies the LL(1) criteria by representing each non-terminal by a function that chooses what functions to invoke by the LL(1) criteria, this sort of parser is called ANSWER.
\begin{itemize}
\item a recursive descent parser
\end{itemize}
\item Programming languages that view programming as describing a step-by-step process to do something are called ANSWER languages.
\begin{itemize}
\item imperative
\end{itemize}
\item Programming languages that view programming as describing characteristics of the problem domain and characteristics of the solution and leaving it to the language processor to find a solution are called ANSWER languages.
\begin{itemize}
\item declarative
\end{itemize}
\item In Prolog, the most natural way to express the fact that `a lion is a cat' is ANSWER.
\begin{itemize}
\item cat(lion).
\item is\_a(lion, cat).
\end{itemize}
\item In Prolog, the most natural way to express the query `what animals are cats?' is ANSWER.
\begin{itemize}
\item animal(What), cat(What).
\item animals(What), cats(What).
\item is\_a(What, animal), is\_a(What, cat).
\item are(What, animals), are(What, cats).
\end{itemize}
\item In Prolog, the most natural way to express the rule that `I am an ancestor of you if I am a parent of you' is ANSWER.
\begin{itemize}
\item ancestor(I, You) :- parent(I, You).
\end{itemize}
\item In Prolog, the most natural way to express the rule that `I am an ancestor of you if I am a parent of an ancestor of you' is ANSWER.
\begin{itemize}
\item ancestor(I, You) :- parent(I, Ancestor), ancestor(Ancestor, You). 
\end{itemize}
\item In Prolog, the expression hi(X, 4) = hi(3, Y) causes X to have the value ANSWER.
\begin{itemize}
\item 3
\end{itemize}
\item In Prolog, the expression hi(X, 4) = hi(3, Y) causes Y to have the value ANSWER.
\begin{itemize}
\item 4
\end{itemize}
\item In Prolog, the expression hi(X, 4) = hi(3, X) causes X to have the value ANSWER.
\begin{itemize}
\item X will not be bound and the expression will fail
\item X will not be bound
\end{itemize}
\item In Prolog, the expression $\lbrack 1, 2, 3\rbrack = \lbrack X | Y\rbrack$ causes X to have the value ANSWER.
\begin{itemize}
\item 1
\end{itemize}
\item In Prolog, the expression $\lbrack 1, 2, 3\rbrack = \lbrack X | Y\rbrack$ causes Y to have the value ANSWER.
\begin{itemize}
\item $\lbrack 2, 3\rbrack$
\end{itemize}
\item In Prolog, the expression $X = \lbrack \lbrack 1,2\rbrack | \lbrack 3,4\rbrack \rbrack$ causes X to have the value ANSWER.
\begin{itemize}
\item $\lbrack \lbrack 1, 2\rbrack, 3, 4\rbrack$
\end{itemize}
\item In Prolog, the expression X = 1 + 2 causes X to have the value ANSWER.
\begin{itemize}
\item 1+2
\end{itemize}
\item In Prolog, the expression 2 = 1 + X causes X to have the value ANSWER.
\begin{itemize}
\item X remains unbound
\item X remains unbound and the expression fails
\end{itemize}
\item In Prolog, the expression that would cause an unbound variable X to take on the sum of the values of a bound variable Y and a bound variable Z is ANSWER.
\begin{itemize}
\item X is Y + Z
\end{itemize}
\item Each named object will have ANSWER, where the name is defined as a synonym for the object.
\begin{itemize}
\item a declaration
\end{itemize}
\item The technical term for connecting a name with an object is ANSWER.
\begin{itemize}
\item binding
\end{itemize}
\item The portion of the program where the name is visible is called its ANSWER.
\begin{itemize}
\item scope
\end{itemize}
\item When the structure of the syntax tree is used to determine which object corresponds to a name, this is called ANSWER.
\begin{itemize}
\item static scoping
\item lexical scoping
\end{itemize}
\item A compiler typically keeps track of which names are associated with which objects by using ANSWER.
\begin{itemize}
\item a symbol table
\item an environment
\end{itemize}
\item ANSWER data structures have the property that no operation on the structure will destroy or modify it.
\begin{itemize}
\item persistent
\item functional
\item immutable
\end{itemize}
\item ANSWER data structures have the property that there are operations on the structure that can destroy or modify it.
\begin{itemize}
\item imperative
\item destructively updated
\item mutable
\end{itemize}
\item Since a compiler may have to look up what object is associated with a name many times, it is typical to use ANSWER to avoid linear search times.
\begin{itemize}
\item hash tables
\end{itemize}
\item In the ICD textbook's example interpreter for evaluating expressions, in the row labelled id, we have the code: v = lookup(vtable, getname(id)) ; if v = unbound then error() else v.  It says getname(id) instead of id, because ANSWER.
\begin{itemize}
\item id indicates a token with a type and value field
\end{itemize}
\item In the ICD textbook's example interpreter for evaluating expressions, in the row labelled id, we have the code: v = lookup(vtable, getname(id)) ; if v = unbound then error() else v.  The value of v would be unbound in the situation that ANSWER.
\begin{itemize}
\item getname(id) was not declared
\item getname(id) was not bound
\end{itemize}
\item In the ICD textbook's example interpreter for evaluating expressions, in the row labelled id(Exps), we have the code: args = EvalExps(Exps,vtable,ftable).  We pass vtable to EvalExps to handle ANSWER.
\begin{itemize}
\item expressions that contain identifiers
\end{itemize}
\item In the ICD textbook's example interpreter for evaluating expressions, in the row labelled id(Exps), we have the code: args = EvalExps(Exps,vtable,ftable).  We pass ftable to EvalExps to handle ANSWER.
\begin{itemize}
\item expressions that contain function usages
\end{itemize}
\item In the ICD textbook's example interpreter for evaluating expressions, in the row labelled let id = Exp1 in Exp2, we have the code: v1 = EvalExp(Exp1, vtable, ftable); vtableP = bind(vtable, getname(id), v1), EvalExp(Exp2, vtableP, ftable).  The bind function changes vtable into vtableP by ANSWER.
\begin{itemize}
\item inserting the association of getname(id) with the value v1 into the table
\item inserting the binding of getname(id) with the value v1 into the table
\end{itemize}
\item Scala was designed to connect two programming paradigms, which were ANSWER.
\begin{itemize}
\item object-oriented and functional
\end{itemize}
\item Another design goal for Scala was to have its programs easily interoperate with those written in ANSWER.
\begin{itemize}
\item Java
\end{itemize}
\item Scala is ANSWER typed
\begin{itemize}
\item statically
\end{itemize}
\item Scala uses few type declarations because its compiler does ANSWER.
\begin{itemize}
\item type inferencing
\end{itemize}
\item The main concurrency method used in Scala is ANSWER.
\begin{itemize}
\item actors
\end{itemize}
\item In Scala, to indicate that a variable is immutable, you introduce it with the ANSWER keyword.
\begin{itemize}
\item val
\end{itemize}
\item In Scala, to indicate that a variable is mutable, you introduce it with the ANSWER keyword.
\begin{itemize}
\item var
\end{itemize}
\item In Scala, if I want to redefine a method that is defined in my parent class, I indicate this by using the keyword ANSWER.
\begin{itemize}
\item override
\end{itemize}
\item The Scala feature closest to a Ruby mixin is the ANSWER.
\begin{itemize}
\item trait
\end{itemize}
\item In Scala, the type that every type is a subtype of is called ANSWER.
\begin{itemize}
\item Any
\end{itemize}
\item In Scala, the type that is a subtype of every type is called ANSWER.
\begin{itemize}
\item Nothing
\end{itemize}
\item Many programming languages represent internal constants for types like strings, floats, and integers.  Scala has the unusual distinction of having an internal constant representation for the type ANSWER, which is normally viewed as a format external to a program.
\begin{itemize}
\item XML
\end{itemize}
\item The ! in Scala is used to ANSWER.
\begin{itemize}
\item send a message to an actor
\end{itemize}
\item In the chapter on Scala, we get the following interesting quote: ANSWER is the most important thing you can do to improve code design for concurrency.
\begin{itemize}
\item Immutability
\end{itemize}
\item In Erlang, the main approach to concurrency is ANSWER.
\begin{itemize}
\item actors
\end{itemize}
\item In the Erlang community, ANSWER code refers to replacing pieces of your application without stopping your application.
\begin{itemize}
\item hot-swapping
\end{itemize}
\item An unusual built-in constant construct in Erlang lets us write $<<4:3,1:3>>$ to represent the value ANSWER.
\begin{itemize}
\item !
\item octal 41
\item decimal 33
\item hexidecimal 21
\end{itemize}
\item Many syntax features of Erlang, such as ending statements with a period, reflect the influence of the programming language ANSWER.
\begin{itemize}
\item Prolog
\end{itemize}
\item In Erlang, you can link two processes together.  Then when one dies, it sends ANSWER to its twin.
\begin{itemize}
\item an exit signal
\end{itemize}
\item The main programming paradigm in Erlang is ANSWER programming.
\begin{itemize}
\item functional
\end{itemize}
\item In Ruby, you would group methods into a class.  In Erlang, you group functions into ANSWER.
\begin{itemize}
\item a module
\end{itemize}
\item The idea that when a process has an error, it is up to a monitoring process to determine what to do about the problem is referred to by the motto ANSWER in Erlang.
\begin{itemize}
\item Let It Crash
\end{itemize}
\item Unlike most Lisp systems, Clojure doesn't use its own custom virtual machine.  It was originally designed to compile to code that would run on the ANSWER.
\begin{itemize}
\item JVM
\item Java Virtual Machine
\end{itemize}
\item The main programming paradigm for Clojure is ANSWER programming.
\begin{itemize}
\item functional
\end{itemize}
\item The loop and recur constructs are in Clojure to guide ANSWER.
\begin{itemize}
\item tail recursion optimization
\item tail recursion elimination
\end{itemize}
\item In Clojure, the value of (repeat 1) is ANSWER.
\begin{itemize}
\item an infinite sequence of 1s
\item a lazy infinite sequence of 1s
\end{itemize}
\item In Clojure, (take 3 (iterate (fn $\lbrack$x$\rbrack$ (* 2 x)) 2)) produces ANSWER.
\begin{itemize}
\item (4 8 16)
\end{itemize}
\item The main Clojure approach to concurrency is called ANSWER.
\begin{itemize}
\item Software Transactional Memory
\item STM
\end{itemize}
\item In Clojure, ANSWER is a concurrency construct that allows an asynchronous return before computation is complete.
\begin{itemize}
\item a future
\end{itemize}
\item In Clojure, you cannot change a reference outside of ANSWER.
\begin{itemize}
\item a transaction
\end{itemize}
\item One approach to speeding up an interpreter is to translate pieces of the code being interpreted directly into machine code during program execution, this is called ANSWER.
\begin{itemize}
\item just-in-time compilation
\end{itemize}
\item The technical term for the compiler design methodology where the translation closely follows the syntax of the language is ANSWER.
\begin{itemize}
\item syntax-directed translation
\end{itemize}
\item Using the straightfoward expression translation scheme in the ICD textbook, if I were to TransExp('3 * x + 1', vtable, ftable), newvar() will be invoked ANSWER times.
\begin{itemize}
\item 5
\end{itemize}
\item Using the straightfoward statement translation scheme in the ICD textbook, if I were to TransStat('if true then z := 1 else z := 2', vtable, ftable), newlabel() will be invoked ANSWER times.
\begin{itemize}
\item 3
\end{itemize}
\item Using the straightfoward statement translation scheme in the ICD textbook, if I were to TransStat('while true do z := 1 + z', vtable, ftable), newlabel() will be invoked ANSWER times.
\begin{itemize}
\item 3
\end{itemize}
\item Using the straightfoward statement translation scheme in the ICD textbook, if I were to TransStat('while z $<$ 3 do z := 1 + z', vtable, ftable), newvar() will be invoked ANSWER times.
\begin{itemize}
\item 5
\end{itemize}
\item When type checking done during program execution, the type system is called ANSWER.
\begin{itemize}
\item dynamic typing
\end{itemize}
\item When type checking done during program compilation, the type system is called ANSWER.
\begin{itemize}
\item static typing
\end{itemize}
\item ANSWER typing is when the language implementation ensures that the arguments of an operation are of the type the operation is defined for.
\begin{itemize}
\item Strong
\end{itemize}
\item ANSWER is the data structure used in language translation to track the binding of variables and functions to their type.
\begin{itemize}
\item A symbol table
\end{itemize}
\item The different traversals of a syntax tree done during compilation associate information with the nodes of the tree.  The technical term for this kind of information is ANSWER.
\begin{itemize}
\item attributes
\end{itemize}
\item ANSWER means that the language allows the same name to be used for different operations over different types.
\begin{itemize}
\item Overloading
\end{itemize}
\item Some languages allow a function to be ANSWER, that is to be defined over a large class of similar types, e.g., over arrays no matter what type their elements are.
\begin{itemize}
\item polymorphic
\item generic
\end{itemize}
\item When a function is invoked, if the language passes a copy of the value of each parameter to the code that performs the function, this is called ANSWER.
\begin{itemize}
\item call-by-value
\item pass-by-value
\end{itemize}
\item If the system stack is used for a call stack, then it becomes important for the caller to update the top of the stack before copying items into it.  The reason is because we are worried about the top of the stack being changed by ANSWER after we have copied in information but before we updated the stack top.
\begin{itemize}
\item an interrupt
\end{itemize}
\item The portion of the call stack associated with a single function invocation and execution is called ANSWER.
\begin{itemize}
\item an activation record
\end{itemize}
\item Another method of parameter passing, whose technical name is ANSWER, is implemented by passing the address of the variable (or whatever the given parameter is).  Assigning to such a parameter would then change the value stored at the address.
\begin{itemize}
\item call-by-reference
\item pass-by-reference
\end{itemize}
\item In C, when you pass a function as a parameter to another function, it is implemented as passing ANSWER.
\begin{itemize}
\item the address of the start of the function code
\end{itemize}
\end{enumerate}
\end{document}
